\documentclass{scrartcl}

\usepackage[utf8]{inputenc}
\usepackage[oldstylenums]{kpfonts}

\RequirePackage[
    backend=biber,
    backref=false,
    style=biblatex-mnras,
    hyperref=true
    ]{biblatex}

\addbibresource{bibliography.bib}  

\RequirePackage[
    colorlinks=true,
    linkcolor=blue,
    citecolor=blue,
    urlcolor=blue,
	]{hyperref} 

\makeatletter
\newcommand\Setmaxbibnames[1]{\renewcommand\blx@maxbibnames{#1}}
\makeatletter

\title{biblatex-mnras}
\author{Fabian Scheuermann\thanks{f.scheuermann@uni-heidelberg.de}}
\date{2025-01-11}

\DeclareUnicodeCharacter{0301}{***************}

\begin{document}

\maketitle

This package implements the bibliography style of the \emph{Monthly Notices of the Royal Astronomical Society} (MNRAS) in \texttt{biblatex}. 
While there is an official \href{https://www.ctan.org/tex-archive/macros/latex/contrib/mnras}{MNRAS document class}, it only works with \texttt{natbib} for the bibliography management. 
The package is intended to be used with the bibentries generated from the \textit{SAO/NASA Astrophysics Data System} (\href{https://ui.adsabs.harvard.edu/}{ADS}). Those entries include the fields \texttt{adsurl}, \texttt{eprint} and \texttt{doi}. They are used to create links to the respective sites which are set behind the journal and volume/page part of the reference. The style can be loaded with
\begin{verbatim}
\usepackage[style=biblatex-mnras]{biblatex}
\end{verbatim}
and for the correct sorting the following code is required before the \verb|\printbibliography|
\begin{verbatim}
\newrefcontext[sorting=nyt]
\end{verbatim}
A copy of this package can be obtained from \href{https://github.com/fschmnn/biblatex-mnras}{GitHub}.

\section*{Style}

The journal asks for an author (year) format and a detailed description can be found in the \href{https://academic.oup.com/mnras/pages/general_instructions?login=false#6.3\%20References\%20and\%20citations}{Instructions to Authors}. 
In the text, the presentation depends on the number of authors:
\begin{itemize}
    \setlength{\itemsep}{-3pt}
    \item One author is either written as \textcite{ArticleOneA} or \parencite{ArticleOneA}.
    \item Two authors use an ampersand \textcite{ArticleTwo} or \parencite{ArticleTwo}. 
    \item Three authors show all three names at first mention \parencite{ArticleThree}, but only the first author thereafter \parencite{ArticleThree}.
    \item Four authors (or more) only show the first author \textcite{ArticleFourA} or \parencite{ArticleFourA}.
    \item Several papers by the same author(s) are styled like \parencite{ArticleOneA,ArticleOneB} or \parencite{ArticleFourB,ArticleFourC}.
    \item When several papers are cited in brackets, they should be ordered by date and separated by semi-colons \parencite{ArticleOneA,ArticleThree,ArticleTwo}.
\end{itemize}
The previous examples were all for articles. Here are some examples for other formats:
\begin{itemize}
    \setlength{\itemsep}{-3pt}
    %arXiv preprint 
    \item A Book looks like \textcite{Book}.
    \item An InBook looks like \textcite{InBook}.
    \item An InProceedings looks like \textcite{InProceedings}.
    \item A Misc looks like \textcite{Misc}.
    \item An Unpublished looks like \textcite{Unpublished}.
\end{itemize}


% https://www.bibtex.com/s/bibliography-style-mnras-mnras/
\defbibnote{myprenote}{This is how the entries look in the bibliography:}
\newrefcontext[sorting=nyt]
\printbibliography[prenote=myprenote,title={References}]

\section*{Issues}

So far, not all entry types are implemented.
In the case of in-proceedings and books, multiple publications from the same author in one year may cause problems. 
The code was only tested with \texttt{biber} as the backend.

\end{document}
