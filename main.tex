\documentclass{scrartcl}

\usepackage[utf8]{inputenc}
\usepackage[oldstylenums]{kpfonts}
\usepackage{enumitem}
\setlist{topsep=0pt, leftmargin=0em,itemsep=0em}

\RequirePackage[
                backend=biber,
                style=biblatex-mnras,
                hyperref=true,
               	uniquename=false,
               	uniquelist=false
                ]{biblatex}

\addbibresource{paper.bib}  % file with references


\RequirePackage{hyperref}  % for links and references

\hypersetup{				% setup the hyperref-package options
	plainpages=false,		% 	-
	linktoc=page,			%	- 
	colorlinks=true,		% 	- colorize links?
	linkcolor=blue,
	citecolor=blue,
	%urlcolor=blue,
	pdfborder={0 0 0},		% 	-
	breaklinks=true,			% 	- allow line break inside links
	bookmarksnumbered=true,	%
	bookmarksopen=true,		%
	bookmarksdepth=2
}

\makeatletter
\newcommand\Setmaxbibnames[1]{\renewcommand\blx@maxbibnames{#1}}
\makeatletter


\title{biblatex-mnras}
\author{Fabian Scheuermann}
\date{June 2022}

% find bad characters
\DeclareUnicodeCharacter{0301}{*************************************}

\begin{document}

\maketitle

based on the \href{https://academic.oup.com/mnras/pages/general_instructions?login=false#6.3\%20References\%20and\%20citations}{Instructions to Authors} from the MNRAS.

\section*{Publications}
\begin{refsection}[publications.bib]
\Setmaxbibnames{1}
\nocite{*}

\printbibliography[heading=none]
\end{refsection}


\newpage
\section{Introduction}
% just some text to showcase how the main bibliography looks
The exact time scale on which HII regions evolve is difficult to pin down. One approach is via the ages of underlying star clusters \parencite[e.g.][]{Whitmore+2011,Hannon+2019,Hannon+2022,Stevance+2020}
, which can be estimated by fitting the observed \emph{spectral energy distribution} (SED) to theoretical models \parencite{Turner+2021}. Another approach is with the help of statistical models \parencite[e.g.][]{Chevance+2020a,Kim+2021}. 
Direct constraints from the ionised nebula on the other hand are rather rare \parencite{Dottori+1981}. One possibility is the H$\alpha$ equivalent width EW(H$\alpha$) \parencite{Copetti+1986,Fernandes+2003,Levesque+2013}. Another possible route is via the H$\alpha$/FUV ratio. Both fluxes are extensively used, most commonly as tracers for star formation \parencite{Hermanowicz+2013,Meurer+2009}, but they have also been used on cloud scales as age indicators \parencite[e.g.][]{SanchezGil+2011,Faesi+2014}. The H$\alpha$ flux is dominated by the most massive stars in the cluster. The FUV and the stellar continuum on the other hand have a significant contribution from lower mass stars. Hence once the most massive stars start to die, both ratios start to decline. This can also be seen in models like \textsc{starburst99} \parencite{Leitherer+2014}, \textsc{bpass} \parencite{Eldridge+2009} or \textsc{cloudy} \parencite{Ferland+2017}, which predict that the ratio decreases monotonically with the age of the cluster.

\printbibliography[title={References}]


\end{document}
